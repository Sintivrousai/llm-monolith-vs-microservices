Η παρούσα εργασία επιδιώκει να συμβάλλει στην αξιολόγηση της χρησιμότητας σύγχρονων LLMs στην ανάπτυξη λογισμικού με έμφαση στην σύγκριση μεταξύ μονολιθικών και microservices αρχιτεκτονικών.
Πιο συγκεκριμένα η συνεισφορά έγκειται στα εξής σημεία:

1) Πρακτική αξιολόγηση LLMs (GPT-4, Code Llama, StarCoder, Codestral) σε πραγματικό περιβάλλον ανάπτυξης REST API εφαρμογής, τόσο με μονολιθική όσο και με microservices προσέγγιση.

2) Συστηματική σύγκριση των μοντέλων με βάση ποσοτικά και ποιοτικά κριτήρια όπως η ακρίβεια του παραγόμενου κώδικα, η ευκολία ενσωμάτωσης και η συχνότητα σφαλμάτων.

3) Ανάδειξη των επιπτώσεων της αρχιτεκτονικής επιλογής στις δυνατότητες και περιορισμούς κάθε μοντέλου, παρέχοντας χρήσιμα συμπεράσματα για μελλοντική αξιοποίηση των LLMs στην ανάπτυξη λογισμικού.

Η εργασία καλύπτει ένα ερευνητικό κενό καθώς δεν υπάρχει εκτενής βιβλιογραφία που να εξετάζει τα LLMs συναρτήσει της αρχιτεκτονικής.

