Η παρούσα εργασία επιδιώκει να συμβάλλει στην αξιολόγηση της χρησιμότητας σύγχρονων LLMs στην ανάπτυξη λογισμικού με έμφαση στην σύγκριση μεταξύ μονολιθικών και microservices αρχιτεκτονικών.
Πιο συγκεκριμένα η συνεισφορά έγκειται στα εξής σημεία:

\begin{enumerate}
    \item Πρακτική αξιολόγηση των LLMs (GPT-4, Claude Sonnet, Devstral, Gemini) σε πραγματικό περιβάλλον ανάπτυξης εφαρμογής τύπου REST API, τόσο με μονολιθική όσο και με microservices αρχιτεκτονική.
    
    \item Συστηματική σύγκριση των μοντέλων με βάση ποσοτικά και ποιοτικά κριτήρια που θα καθοριστούν στην συνέχεια.
    
    \item Ανάδειξη των επιπτώσεων της αρχιτεκτονικής επιλογής (μονολιθική vs. microservices) στις δυνατότητες και τους περιορισμούς κάθε μοντέλου, προσφέροντας χρήσιμα συμπεράσματα για τη μελλοντική αξιοποίηση των LLMs στην ανάπτυξη λογισμικού.
\end{enumerate}

\vspace{1em}

Η εργασία καλύπτει ένα ερευνητικό κενό καθώς δεν υπάρχει εκτενής βιβλιογραφία που να εξετάζει τα LLMs συναρτήσει της αρχιτεκτονικής.




