Τα τελευταία χρόνια, η ραγδαία ανάπτυξη των Μεγάλων Γλωσσικών Μοντέλων (LLMs) έχει ανοίξει νέους δρόμους στην ανάπτυξη λογισμικού, προσφέροντας εργαλεία ικανά να παράγουν, να βελτιστοποιούν και να ενσωματώνουν κώδικα με υψηλή ακρίβεια και ταχύτητα. Μοντέλα όπως το GPT-4 και το Code Llama έχουν αναδειχθεί ως ισχυροί συνεργάτες των προγραμματιστών, ενισχύοντας την αποδοτικότητα και την παραγωγικότητα τους σε ποικίλα έργα ανάπτυξης λογισμικού.

Παράλληλα, η επιλογή αρχιτεκτονικής αποτελεί κρίσιμο ζήτημα στον σχεδιασμό εφαρμογών, με τις μονολιθικές και microservices αρχιτεκτονικές να κυριαρχούν ως οι δύο βασικές προσεγγίσεις. Η μονολιθική αρχιτεκτονική, με την ενιαία δομή της, προσφέρει απλότητα και άμεση διαχείριση, ενώ η microservices αρχιτεκτονική επιτρέπει την αποκέντρωση και την ευελιξία μέσω ανεξάρτητων υπηρεσιών.

Η παρούσα μελέτη εστιάζει στη συγκριτική αξιολόγηση των LLMs στην ανάπτυξη μιας REST API εφαρμογής σε Java, η οποία θα υλοποιηθεί τόσο σε μονολιθική όσο και σε microservices αρχιτεκτονική. Στόχος είναι να αναλυθούν οι δυνατότητες των LLMs στην παραγωγή κώδικα και να διερευνηθεί κατά πόσο η αρχιτεκτονική επηρεάζει την απόδοσή τους. Για τη σύγκριση θα ληφθούν υπόψη παράγοντες όπως η ποιότητα του παραγόμενου κώδικα, η αποδοτικότητα της εφαρμογής και η διαχείριση σφαλμάτων, ώστε να εξαχθούν χρήσιμα συμπεράσματα για την αποτελεσματικότητα κάθε μοντέλου σε διαφορετικά αρχιτεκτονικά περιβάλλοντα.