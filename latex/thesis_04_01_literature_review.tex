\subsection{Τι είναι τα Μεγάλα Γλωσσικά Μοντέλα;}
Τα μεγάλα γλωσσικά μοντέλα (Large Language Models - LLMs) είναι προηγμένα συστήματα τεχνητής νοημοσύνης, ικανά να κατανοήσουν και να παράξουν φυσική γλώσσα με ευκολία, ακρίβεια και φυσικότητα. Τα LLMs έχουν προεκπαιδευτεί σε τεράστιους όγκους κειμένων, προερχόμενων από διάφορες πηγές, όπως βιβλία, άρθρα και διαλόγους, χρησιμοποιώντας τεχνικές μηχανικής μάθησης και ειδικότερα βαθιάς μάθησης (deep learning). Οι δυνατότητές τους είναι αρκετά εξελιγμένες καθώς μπορούν να απαντούν φυσικά ερωτήσεις, να συντάσουν πλήρη κείμενα, να δημιουργούν περιλήψεις, να καταλαβαίνουν και να προσαρμόζουν το ύφος του κειμένου. 

Όσον αφορά τον τομέα του προγραμματισμού, πλέον, αποτελούν σημαντικά εργαλεία στα χέρια των προγραμματιστών. Μερικές από τις λειτουργίες που προσφέρουν είναι η παραγωγή κώδικα, η προτάσεις βελτιστοποίησης, η επεξήγηση του, η εύρεση και διόρθωση σφαλμάτων σε ποικίλες γλώσσες προγραμματισμού. Ωστόσο, παρά τις δυνατότητές τους, τα μοντέλα αυτά έχουν και ορισμένους περιορισμούς. Ο παραγόμενος κώδικας μπορεί να περιέχει σφάλματα ή μη βέλτιστες λύσεις, ιδιαίτερα σε σύνθετες ή εξειδικευμένες περιπτώσεις. Επιπλέον, δεν έχουν πραγματική «κατανόηση» του προβλήματος ή των απαιτήσεων του έργου και βασίζονται αποκλειστικά σε πρότυπα που έχουν μάθει από τα δεδομένα εκπαίδευσής τους. Για τον λόγο αυτό, η ανθρώπινη επίβλεψη και κριτική σκέψη παραμένουν απαραίτητες.

\subsection{Ιστορική Αναδρομή}
Αρχικά, τα LLMs αποτέλεσαν μια προσπάθεια της τεχνητής νοημοσύνης να επεξεργαστούν την φυσική γλώσσα, βασισιζόμενοι στην στατιστική. Στην πορεία, εμφανίστηκαν οι όροι βαθιά μάθηση και νευρωνικά δίκτυα που αποτέλεσαν την πραγματική πρόοδο τους. Συγκεκριμένα, τα νευρωνικά δίκτυα είναι υπολογιστικά μοντέλα που μιμούνται τον τρόπο που λειτουργεί ο ανθρώπινος εγκέφαλος. Η βαθιά μάθηση είναι κλάδος της μηχανικής μάθησης και χρησιμοποιεί τεχνητά νευρωνικά δίκτυα πολλών στρώσεων (layers) για να επεξεργαστεί την πληροφορία που λαμβάνει σε πολλά επίπεδα.

Η εξέλιξη των LLMs ξεκίνησε δυναμικά το 2013, όπου η Google ανέπτυξε το μοντέλο Word2Vec, το οποίο αναπαραστούσε την σημασία των λέξεων με βάση τα συμφραζόμενα, μέσω διανυσματικών αναπαραστάσεων. Το 2018, η Google ανέπτυξε το μοντέλο του BERT (Bidirectional Encoder Representations from Transformers), που θεωρήθηκε επαναστατικό καθώς μπορούσε να διαβάσει λέξεις και από τις δύο κατευθύνσεις. Αυτή η δυνατότητα αύξησε ραγδαία την κατανόηση της φυσικής γλώσσας και έθεσε νέα θεμέλεια στον κλάδο. Την ίδια χρονιά, η OpenAI ανέπτυξε το GPT (Generative Pre-trained Transformer), το οποίο βασίστηκε στους μετασχηματιστές  και ήταν πρωτοπόρο, εισάγοντας την έννοια της μαζικής προεκπαίδευσης σε μεγάλο όγκο δεδομένων. Τα επόμενα έτη, η OpenAI προχώρησε σε νέες εκδόσεις του όπως GPT-2, GPT-3, GPT-3.5, με το GPT-4 να αποτελεί την πιο εξελιγμένη του έκδοση.

Τα παραπάνω μοντέλα αποτέλεσαν ορόσημα για τον κλάδο, ο οποίος έχει εξελιχθεί ραγδαία τα τελευταία χρόνια, με νέα μοντέλα και καινοτόμες λειτουργίες να αναπτύσσονται συνεχώς. Στην παρούσα φάση, πολλές εταιρείες παγκοσμίως επικεντρώνονται στην ανάπτυξη μοντέλων που ασχολούνται με προγραμματιστικές λειτουργίες όπως η κατανόηση και συγγραφή κώδικα που προαναφέρθηκαν. Κάποια από αυτά τα μοντέλα είναι το GPT-4, το Claude Sonnet, το Gemini και το Devstral, τα οποία θα μελετηθούν στην συνέχεια της εργασίας.

\subsection{GPT-4}
Το GPT-4(Generative Pre-trained Transformer 4)  αναπτύχθηκε από την OpenAI και κυκλοφόρησε το 2023. Αποτελεί ένα από τα πιο προηγμένα συστήματα LLM, καθώς είναι η εξέλιξη του GPT-3, με αρκετά βελτιωμένες λειτουργίες στην κατανόηση και παραγωγή φυσικής γλώσσας. Χρησιμοποιείται ευρέως από την πλατφόρμα ChatGPT, αλλά και σε πολλές άλλες εφαρμογές και περιβάλλοντα ανάπτυξης λογισμικού, όπου μπορεί να παράγει, να εξηγεί και να διορθώνει κώδικα σε διάφορες γλώσσες, όπως Python, JavaScript, C++, HTML και άλλες. Επιπλέον είναι σε θέση να αναλύει απαιτήσεις, να προτείνει λύσεις και βελτιστοποιήσεις και να εντοπίζει λάθη. 

\subsection{Code Llama}
Το  Code Llama αναπτύχθηκε από την Meta το 2023, και ανήκει στην οικογένεια μοντέλων Llama (Large Language Model Meta AI), η οποία βασίζεται στην τεχνολογία των μετασχηματιστών (transformers) όπως και το GPT. Είναι εξειδικευμένο εργαλείο, σχεδιασμένο κυρίως για χρήση σε προγραμματιστικά περιβαλλοντα, αν και μπορει να απαντήσει και σε ποιο γενικές ερωτήσεις. Το μοντέλο υποστηρίζει πολλές γλώσσες προγραμματισμού και είναι ικανό να κατανοεί, να παράγει και να βελτιστοποιεί κώδικα με υψηλή ακρίβεια. Η Meta έχει διαθέσει το μοντέλο υπό μια ανοικτή άδεια χρήσης, επιτρέποντας την αξιοποίηση του χωρίς κόστος σε περιβάλλοντα ανάπτυξης λογισμικού, γεγονός που το καθιστά ιδιαίτερα ελκυστικό στον κλάδο.

\subsection{StarCoder}
Το StarCoder αναπτύχθηκε από την συνεργασία των Hugging Face και ServiceNow Research το 2023, στο πλαίσιο του έργου BigCode. Αποτελεί ένα μοντέλο ανοιχτού κώδικα, που εξειδικεύεται σε προγραμματιστικές λειτουργίες. Το μοντέλο εκπαιδεύτηκε σε έναν μεγάλο όγκο δημόσιων αποθετηρίων του Github, με σκοπό την καλύτερη κατανόηση και παραγωγή κώδικα. Ένα από τα βασικά χαρακτηριστικά που περιλαμβάνει είναι οι μηχανισμοι δεοντολογικού ελέγχου (ethical filtering), οι οποίοι περιορίζουν την χρήση ευαίσθητων δεδομένων. Ωστόσο, αξίζει να σημειωθεί ότι η προεκπαίδευση του μοντέλου με χρήση δημόσιων αποθετηρίων μπορεί να επηρεάσει τις λύσεις που θα προτείνει, καθώς το Github περιλαμβάνει ενα ευρύ φάσμα αποθετηρίων από πολύ ποιοτικές υλοποιήσεις μέχρι και αναποτελεσματικές ή λανθασμένες. Λογικό επακόλουθο είναι, το StarCoder, να μην προτείνει πάντοτε τις βέλτιστες λύσεις. Βέβαια, εξακολουθεί να αποτελεί ένα πολύ ικανό εργαλείο για όποιον αναζητά ένα διαφανές και ισχυρό εργαλείο ανοιχτού κώδικα.

\subsection{Codestral}
Το Codestral αναπτύχθηκε από την Mistral AI, to 2024 και αξιοποιεί τους μετασχηματιστές, στο ίδιο μήκος κύματος με το GPT και το Code Llama. Έχει εκπαιδευτεί με βασικό στόχο την κατανόηση και παραγωγή κώδικα, και ειδικεύεται σε προγραμματιστικές εργασίες. Διατίθεται με ανοικτή άδεια χρήσης που επιτρέπει την χρήση του τόσο σε ερευνητικά περιβάλλοντα όσο και σε περιβάλλοντα ανάπτυξης λογισμικού. Ωστόσο η άδεια χρήσης του απαγορεύει την ενσωμάτωσή του σε κλειστού τύπου (black box) εφαρμογές και σε λογισμικό που τρέχει τοπικά σε συσκευές των χρηστών. Αυτοί οι περιορισμοί διαφοροποιούν το Codestral από άλλα μοντέλα ανοιχτού κώδικα. Σε γενικό πλαίσιο, το μοντέλο παρουσιάζει αρκετά καλές επιδόσεις στον τομέα της ανάπτυξης λογισμικού αποτελόντας ένα πολλά υποσχόμενο εργαλείο για τους προγραμματιστές.

\subsection{Gemini}

\subsection{Devstral}

\subsection{Claude Sonnet}
