\subsection{Εισαγωγή}
Η αρχιτεκτονική του λογισμικού αποτελεί καθοριστικό παράγοντα στην δομή ενός συστήματος καθώς και κρίσιμο παράγοντα της επιτυχίας του,  ιδίως σε ό,τι αφορά την επεκτασιμότητα, τη συντηρησιμότητα, την ευκολία ανάπτυξης και τη δυνατότητα αυτοματοποίησης. Καθώς η ανάπτυξη λογισμικού εξελίσσεται, νέες τάσεις, ανάγκες και τεχνολογικά εργαλεία οδηγούν στην εμφάνιση και στην υιοθέτηση διαφορετικών αρχιτεκτονικών προσεγγίσεων. Στο πλαίσιο αυτής της εργασίας, εστιάζουμε στη συγκριτική μελέτη δύο κυρίαρχων προσεγγίσεων: της μονολιθικής και της microservices αρχιτεκτονικής, ειδικά στο περιβάλλον ανάπτυξης REST API σε Java.

\subsection{Μονολιθική Αρχιτεκτονική}
Η μονολιθική αρχιτεκτονική αποτελεί την πιο ιστορική και παραδοσιακή προσέγγιση της ανάπτυξης λογισμικού. Η βασική ιδέα βρίσκεται στην συγκέντρωση όλων τον συστατικών μιας εφαρμογής (Frontend, Backend, πρόσβαση στην βάση δεδομένων, κα) σε ένα ενιαίο έργο.

Ένα από τα βασικά της πλεονεκτήματα είναι η απλότητα στην ανάπτυξη, τη δοκιμή και την υλοποίηση, καθώς όλα τα υποσυστήματα βρίσκονται στον ίδιο χώρο εκτέλεσης, επιτρέποντας άμεση επικοινωνία μεταξύ τους. Αυτό καθιστά ευκολότερη την εύρεση σφαλμάτων και την ενσωμάτωση αλλαγών. Παράλληλα, η μονολιθική προσέγγιση συχνά έχει καλύτερες επιδόσεις, επειδή όλα τα μέρη της εφαρμογής επικοινωνούν μεταξύ τους απευθείας, χωρίς να χρειάζεται να επικοινωνούν μέσω δικτύου όπως συμβαίνει σε πιο κατανεμημένα συστήματα.

Ωστόσο, η μονολιθική δομή ενέχει σημαντικά μειονεκτήματα, ειδικά σε εφαρμογές μεγάλης κλίμακας ή ταχείας εξέλιξης. Καθώς το σύστημα μεγαλώνει, καθίσταται δύσκολη η συντήρηση, ενώ ακόμα και μικρές αλλαγές ενδέχεται να επηρεάσουν απρόβλεπτα άλλα μέρη της εφαρμογής. Επιπρόσθετα, ο περιορισμένος βαθμός επεκτασιμότητας, η δυσκολία στην υιοθέτηση νέων τεχνολογιών ανά υποσύστημα, καθώς και η αδυναμία ανεξάρτητης ανάπτυξης λειτουργιών αποτελούν σημαντικούς περιορισμούς της ευελιξίας της αρχιτεκτονικής.

\subsection{Microservices Αρχιτεκτονική} 
Η αρχιτεκτονική microservices βασίζεται στη λογική της διάσπασης μιας εφαρμογής σε μικρότερες, αυτόνομες υπηρεσίες, καθεμία από τις οποίες εκτελεί μια συγκεκριμένη λειτουργία και επικοινωνεί με τις υπόλοιπες μέσω κατάλληλων διεπαφών (όπως REST APIs). 

Αυτή η προσέγγιση προσφέρει σημαντικά πλεονεκτήματα, όπως αυξημένη ευελιξία και ευκολία στην ανάπτυξη και συντήρηση του λογισμικού. Κάθε υπηρεσία μπορεί να αναπτυχθεί, να δοκιμαστεί και να βελτιστοποιηθεί ανεξάρτητα, κάτι που επιταχύνει την κυκλοφορία νέων χαρακτηριστικών. Επίσης, διευκολύνει την κλιμάκωση της εφαρμογής, καθώς μπορούν να αυξηθούν οι πόροι μόνο για τις υπηρεσίες που το απαιτούν. 

Ωστόσο, η προσέγγιση αυτή εισάγει και νέες προκλήσεις. Η πολυπλοκότητα της υποδομής αυξάνεται, απαιτώντας λύσεις για τη διαχείριση της επικοινωνίας, του συντονισμού και της ασφάλειας μεταξύ των υπηρεσιών. Επιπλέον, η ύπαρξη πολλών διαφορετικών υπηρεσιών συνεπάγεται συχνά μεγαλύτερη ανάγκη για παρακολούθηση, logging και debugging, γεγονός που μπορεί να αυξήσει το κόστος και τον χρόνο ανάπτυξης. Παρά τις δυσκολίες, τα microservices παραμένουν ιδιαίτερα δημοφιλή σε μεγάλες εφαρμογές και οργανισμούς που επιζητούν ευελιξία και δυνατότητα συνεχούς ανάπτυξης.

\subsection{Συγκριτική ανάλυση στο Πλαίσιο REST API σε Java}
Τα REST APIs (Representational State Transfer Application Programming Interfaces) αποτελούν έναν διαδεδομένο και αρκετά δημοφιλή τρόπο επικοινωνίας στα συστήματα λογισμικού. Η επικοινωνία γίνεται μέσω του πρωτοκόλλου HTTP (Hypertext Transfer Protocol), επιτρέποντας την αποστολή και λήψη δεδομένων μέσω καθοριμένων διαδρομών (endpoints). Σε περιβάλλοντα της Java, για την ανάπτυξη Rest Apis εφαρμογών, συχνά, επιλέγεται η χρήση του framework Spring Boot, καθώς περιέχει δομημένες μεθόδους για καλύτερη διαχείρηση των αιτημάτων. Η αρχιτεκτονική που επιλέγεται (μονολιθική ή microservices) επηρεάζει σημαντικά αρχικά την κατασκευή και μετέπειτα τις δυνατότητες ελέγχου, συντήρησης και επέκτασης.

\textbf{Μονολιθική Προσέγγιση}: Η δημιουργία όλων των endpoints πραγματοποιείται σε ένα ενιαίο αρχείο που περιλβάνει όλα τα συστατικά όπως τις οντότητες (entities), υπηρεσίες (services) και τους ελεγκτές (controllers). 

\textbf{Microservices Προσέγγιση}: Κάθε επιμέρους λειτουργηκότητα αποτελεί μια ξεχωσριστή υπηρεσία (service) με δικό της REST API. Η επικοινωνία των υπηρεσιών αυτών συντελέιται μέσω HTTP  και απαιτούνται κάποιοι μηχανισμοί συντονισμού.

Παρακάτω ακολουθεί ένας συνοπτικός πίνακας με τα πλεονεκτήματα και τα μειονεκτήματα της κάθε προσέγγισης.

\begin{table}[H]
\centering
\caption{Σύγκριση Μονολιθικής και Microservices Αρχιτεκτονικής σε REST API με Java}
\begin{tabularx}{\textwidth}{|X|X|X|}
\hline
\textbf{Κριτήριο} & \textbf{Μονολιθική Αρχιτεκτονική} & \textbf{Microservices Αρχιτεκτονική} \\
\hline
Απλότητα Ανάπτυξης & Υψηλή & Χαμηλότερη  \\
\hline
Επεκτασιμότητα & Περιορισμένη & Υψηλή \\
\hline
Συντηρησιμότητα & Δύσκολη σε μεγάλα έργα & Ευκολότερη  \\
\hline
Ελεγκσιμότητα (Testability) & Περιορισμένη & Υψηλή  \\
\hline
Απόδοση & Υψηλή & Πιο αργή \\
\hline
Τεχνολογική Ευελιξία & Περιορισμένη & Υψηλή  \\
\hline
Δυσκολία Υλοποίησης & Χαμηλή & Υψηλή \\
\hline
Χρήση σε Μικρές Εφαρμογές & Κατάλληλη & Υπερβολική πολυπλοκότητα \\
\hline
\end{tabularx}
\end{table}


\subsection{Συσχέτιση με LLMs}
Οι δυνατότητες των LLMs διαφοροποιούνται ανάλογα με την αρχιτεκτονική που καλούνται να υλοποιήσουν. Από την μία, η μονολιθική αρχιτεκτονική διευκολύνει τις οδηγίες που θα δώσουμε στο μοντέλο, λόγω του συνεκτικού της χαρακτήρα. Ένα LLM μπορεί να κατανοήσει καλύτερα τις απαιτήσεις και να παράξει καλύτερα αποτελέσματα όταν έχει εικόνα όλου του έργου.

Αντιθέτως, όταν ένα σύστημα αναπτύσσεται με microservices, κάθε υπηρεσία είναι αυτόνομη και ανεξάρτητη. Τα LLMs αποδίδουν καλά στην ανάπτυξη κάθε επιμέρους υπηρεσίας καθώς αποτελούν μικρότερα και πιο στοχευμένα τμήματα του συστήματος. Παράλληλα, τέτοια μοντέλα παρέχουν σημαντική βοήθεια στην τεκμηρίωση, τους ελέγχους και στις βελτιστοποιήσεις του κώδικα, γεγονός αρκετά σημαντικό στην περίπτωση των microservices λόγω αυξημένης πολυπλοκότητας.

Σε γενικό πλαίσιο, τα LLMs μπορούν να προσαρμοστούν και στις δύο προσεγγίσεις, με την αποτελεσματικότητά τους να διαφέρει βεβαίως. Σε πιο μικρές μονολιθικές εφαρμογές μπορούν να συνδράμουν στην ολιστική παραγωγή του κώδικα ενώ στις microservices βοηθούν στην επιμέρους παραγωγή των αυτόνομων μονάδων.