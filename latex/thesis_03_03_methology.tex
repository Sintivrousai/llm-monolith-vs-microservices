Η μεθοδολογία που ακολουθείται περιλαμβάνει μια εκτενής βιβλιογραφική επισκόπηση σχετικά με το θεωρητικό υπόβαθρο. Η μελέτη, ωστόσο, θα επικεντρωθεί στο πρακτικό κομμάτι. Τα βήματα που θα ακολουθηθούν είναι τα εξής:

1) Επιλογή των LLMs προς αξιολόγηση: Επιλέγονται τέσσερα σύγχρονα και ευρέως χρησιμοποιούμενα μοντέλα τα οποία είναι το GPT-4, το Claude Sonnet, το Gemini και το Devstral.

2) Καθορισμός μιας απλής και συνάμα λειτουργικής εφαρμογής σε Java που εκτελεί κάποιες βασικές λειτουργίες.

3) Υλοποίηση της εφαρμογής: Η ιδια εφαρμογή υλοποιείται με 2 εκδοχές, μια για κάθε αρχιτεκτονική για κάθε LLM.

4) Αξιολόγηση απόδοσης των μοντέλων: 
Τα μοντέλα συγκρίνονται βάσει των εξής κριτηρίων:

\begin{itemize}
    \item \textbf{Πληρότητα Λύσης}: έλεγχος αν ο κώδικας που παρήγαγε το μοντέλο πληροί όλες τις προϋποθέσεις και δεν έχει κάποια σημαντική παράληψη κλάσης, εξάρτησης ή configuration
    \item \textbf{Ορθότητα κώδικα}: μέτρηση συντακτικών λαθών και λαθών χρόνουη εκτέλεσης
    \item \textbf{Κατανόηση αρχιτεκτονικής}: έλεγχος αν το μοντέλο κατανόησε και ακολούθησε την αρχιτεκτονική που του ζητήθηκε
    \item \textbf{Δομή}: έλεγχος αν είναι σαφής ο διαχωρισμός των αρμοδιοτήτων (services, controllers, κα) σε διαφορετικές κλάσεις
    \item \textbf{Αναγνωσιμότητα και καθαρότητα}: έλεγχος ονομάτων κλάσεων και μεθόδων, ύπαρξη επεξηγηματικών σχολίων, τήρηση checkstyle
    \item \textbf{Debugging}: πόσο εύκολα αναγνώρισε το μοντέλο το ακριβές σημείο του λάθους και πόσο; χρόνος χρειάστηκε για να το επιλύσει
    \item \textbf{Χρόνος υλοποίησης}: μέτρηση χρόνου απόκρισης του μοντέλου, ενσωμάτωσης σε IDE (Intellij), εκτέλεσης και debugging
    \item \textbf{Στατική ανάλυση}: Χρήση του εργαλείου SonarQube για μετρήσεις λαθών και πολυπλοκότητας
    \item \textbf{Κόστος} χρήσης των μοντέλων
\end{itemize}

5) Καταγραφη, ανάλυση και σύγκριση αποτελεσμάτων

6) Διατύπωση συμπερασμάτων και προτάσεων
