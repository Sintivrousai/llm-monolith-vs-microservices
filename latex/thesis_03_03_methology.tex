Η μεθοδολογία που ακολουθείται περιλαμβάνει μια εκτενής βιβλιογραφική επισκόπηση σχετικά με το θεωρητικό υπόβαθρο. Η μελέτη, ωστόσο, θα επικεντρωθεί στο πρακτικό κομμάτι. Τα βήματα που θα ακολουθηθούν είναι τα εξής:

1) Επιλογή των LLMs προς αξιολόγηση: Επιλέγονται τέσσερα σύγχρονα και ευρέως χρησιμοποιούμενα μοντέλα τα οποία είναι το GPT-4, το Code Llama, το StarCoder και το Codestral.

2) Καθορισμός μιας απλής και συνάμα λειτουργικής εφαρμογής σε Java που εκτελεί κάποιες βασικές λειτουργίες.

3) Υλοποίηση της εφαρμογής: Η ιδια εφαρμογή υλοποιείται με 2 εκδοχές, μια για κάθε αρχιτεκτονική για κάθε LLM.

4) Αξιολόγηση απόδοσης των μοντέλων: 
Τα μοντέλα συγκρίνονται βάσει των εξής κριτηρίων:

\begin{itemize}
    \item Ακρίβεια παραγόμενου κώδικα
    \item Ευκολία ενσωμάτωσης και εκτέλεσης
    \item Λάθη ή δυσλειτουργίες
    \item Απαιτήσεις σε ανθρώπινη παρέμβαση (prompt refinement, debugging)
\end{itemize}

5) Καταγραφη, ανάλυση και σύγκριση αποτελεσμάτων

6) Διατύπωση συμπερασμάτων και προτάσεων
